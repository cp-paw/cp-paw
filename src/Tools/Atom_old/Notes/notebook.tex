\section{Remarks} 
\begin{itemize}
\item In a test for hydrogen we found that the plane-wave
converged result deviates if the core radius was small. It turned out
that the vhat truncation error, which is estimated and printed in the
unscreening routine, became large. The solution was to increase the
number of projector functions.

In that case the projector constructed from and acting on a bound
state, differed from unity.

\item the compensation charge density may be a problem. A suggestion
  is to fix the decay to a fourth of the covalent radius. In that case
  the density at the covalent radius is about 10$^-7$ times the number
  of valence electrons.

\item A d-partial wave component is important even for first-row double
  bonded species.

\item technical problems with the atomc program: K, Sc-Ti, Ni-Ge,
  Se-Br, Rb, Y-Zr, Rh, Cs, Ba-Ta, Au-Hg, Fr-Md, Lr-106

\item using $\lambda=6$ for the construction of the PS-partialwaves,
  results in projector functions that extend to $1.5 r_c$. It is
  instructive to look at the projector functions multiplied with
  $r^2$, in order to see, which area is relevant for the augmentation.
  
\item the increase of projector functions and the unscreening step are
  not exactly commutative. However, they should be commutative, if the
  minimum augmentation is constructued from the occupied states. In
  that case the contribution of higher projectors should be
  neglegible.
  
\item the size of projector functions and partial waves is not well
  balanced. The projector functions appear to increase in steps of
  about an order of magnitude, while the partial waves appear to
  decrease my a similar amount. A better choice would be obtained if
  $r^2\tilde{p}$ is normalized to unity.

\end{itemize}

\section{Radial grids}
All calculation inthe atomic program are done on radial logarithmic
grids of the form
\begin{equation}
r_i=r_1{rm e}^{\alpha(i-1)}
\end{equation}
Typical values are $r1=1.024\times10^{-4}$ and $\alpha=1/20$. We use
250 grid points.



\section{All-electron atomic calculation}
We construct a starting potential using the Thomas Fermi expression
which correspons to the minimum of the two potentials:
\begin{eqnarray*}
x&=&r/\sqrt[3]{0.69395656/Z}\\
v_s({\bf r})&=&-\frac{Z}{r} \frac{1}{1+\sqrt{x}(0.0274-x*(0.1486-0.007298*x))
+x*(1.24x(0.2302+0.006944x))}\\
v({\bf r})&=& -Z/r
end{eqnarray}
Note that this is only the radial part and needs to be multiplied by $Y_s$.

Next we determine a first guess for the energy eigenvalues assuming
for each state a Coulomb-potential with an effective charge which has an
in absolute value corresponding to the total number of electrons in
the actual shell and all lower lying shells, minus one electron.
\begin{displaymath}
\epsilon_n=-\frac{1}{2} (\frac{Z^*_{\ell,n}}{n})^2
\end{displaymath}
Here $n=\ell+#(nodes)+1$ is the main quantum number. As an example a
hydogen atom level would be predicted with $Z^*_{0,1}=1$. If the such
predicted one-particle level lies above the potential tail, we
shift it 0.1~H below the tail. This has been done to avoid starting
with scattering states.

Now we perform fixed node calculation for every one-particle state,
add the charge density, potential. The input and output potentials are
mixed and ythe process is repeated until selfconsistency.

Next the total energy contributions are calculated. The core density,
potential, the one-particle wave functions and energies are kept.

\subsection{Fixed-node calculation}
\subsection{Potential calculation}

The potential is calculated using the Poisson Solver ({\tt
  radial\$poisson} and DFT 
functional module from the simulation code. 
The derivative for the gradient corrections is
 obtained using the {\tt radial\$derive} function of the radial module.

\begin{eqnarray*}
v(r)&=&-Z/r+\int\rho{|r-r^\prime}+\frac{\partial E_{xc}}{\partial \rho}
-\Bigl(\frac{2}{r}+\frac{partial}{\partial_r}\Bigr)
\Bigl[\frac{E_{xc}}{\partial(\partial\rho/\partial r)}\Bigr]
\\
\frac{E_{xc}}{\partial(\partial\rho/\partial r)}
&=&2\frac{\partial\rho}{\partial r} \frac{E_{xc}}{\partial(\partial\rho/\partial r)^2}
\end{eqnarray*}

\subsection{Potential mixing}

The potentials are mixed using Andersen mixing.
\begin{equation}
v_{i,+}(r)=
\end{equation}

\section{Pseudo-potential}
\section{Pseudo-core}
\section{Pseudo-partial waves}


%===============================================================
\section{Projector functions}
%===============================================================

%===============================================================
\subsection{Preliminary projector functions}
%===============================================================
We normalize partial waves such that that the AE-partialwaves 
are normalized to unity within the augmentation radius.

Then we calculate 
\begin{eqnarray*}
dT&=&-\frac{1}{2}\Bigl[\langle\phi|\nabla^2|\phi\rangle
-\langle\tilde\phi|\nabla^2|\tilde\phi\rangle\Bigr]
\\
dH&=&dT+\Bigl[\langle\phi|v|\phi\rangle
-\langle\tilde\phi|\tilde{v}|\tilde\phi\rangle\Bigr]
\\
dO&=&\Bigl[\langle\phi|\phi\rangle
-\langle\tilde\phi|\tilde\phi\rangle\Bigr]
\end{eqnarray*}
Note that $\nabla^2|\phi\rangle$ is obtained as
$2(v-\epsilon)|phi\rangle$ using the potential from which the partial
waves have been created, and therefore the Laplacian is not exactly
the laplacian, but also contains relativistic effects. The
relativistic effects correspond to the Laplacian contribution on the
small component of the wave function.

Now we construct the projector functions
\begin{equation}
\tilde{p}_i=(-\frac{1}{2}+\tilde{v}-\epsilon_i)|\tilde\phi_i\rangle
\end{equation}
The projector function obtained this way is truncated beyond the
augmentation radius.

If
$\langle\tilde{p}|\tilde\phi\rangle/\langle\tilde\phi|\tilde\phi\rangle$
is smaller than a tolerance (10^{-5}) we decide that the projector is
orthogonal to the partial wave. In that case it is currecntly replaced
by $e^{-(r/r_\Omega)^6}\phi(r)$. This is only a fix and should be
replaced.
%
%===============================================================
\subsection{Bi-orthogonality $\langle\tilde{p}_i|\tilde\phi_j\rangle
=\delta_{i,j}$}
%===============================================================

Now we need to enforce the bi-orthonormality condition 
$\langle\tilde{p}_i|\tilde\phi_j\rangle=\delta_{i,j}$. 
\begin{enumerate}
\item we make the PS-partial waves orthogonal to lower projector
  functions
 \begin{displaymath}
   |\tilde\phi_i\rangle&=&|\tilde\phi_i\rangle-\sum_j^{i-1}
|\tilde\phi_j\rangle\langle\tilde{p}_j|\tilde\phi_i\rangle
\end{displaymath}
For each term, we correct the one-center matrices accordingly.
\item we make the projector function orthogonal to lower projector
  functions
 \begin{displaymath}
   |\tilde{p}\rangle&=&|\tilde{p}_i\rangle-\sum_j^{i-1}
|\tilde{p}\rangle\langle\tilde\phi_j|\tilde{p}_i\rangle
\end{displaymath}
\item we normalize the projector function
 \begin{displaymath}
   |\tilde{p}\rangle&=&|\tilde{p}_i\rangle/\langle\tilde\phi_i|\tilde{p}_i\rangle
\end{displaymath}
\item Finally, as an in principle unneccesary step, which is included
  to avoid numerical instabilities, we normalize the partial waves
  such that the norm of the AE partial waves within the augmentation
  sphere is unity. In future, the {\it generalized} atomic sphere shall
  be taken instead of the augmentation sphere.
 \begin{eqnarray*}
   |\phi_i\rangle&=&|\phi_i_i\rangle
     /\sqrt{\langle\phi_i|\theta_\Omega|\phi_i\rangle}
\\
   |\tilde\phi_i\rangle&=&|\tilde\phi_i_i\rangle
     /\sqrt{\langle\phi_i|\theta_\Omega|\phi_i\rangle}
\\
   |\tilde{p}\rangle&=&|\tilde{p}_i\rangle
    \sqrt{\langle\phi_i|\theta_\Omega|\phi_i\rangle}
\end{eqnarray*}
we correct the one-center matrices accordingly.
\end{enumerate}
%
%===============================================================
\section{Unscreening and $\hat{v}$}
%===============================================================
%
%===============================================================
\section{Testing plane wave convergence}
%===============================================================
%
The plane wave convergence is estimated from the one-particle wave
functions of a non-self-consistent PAW calculation using 
potentials $\tilde{v}$ and $\tilde{H}^2$. Per
construction, these potentials should be identical to the
self-consistent potentials, but may not cover some numerical
errors. 

The wave functions are transformed into Fourier-space using
Besseltransforms. 

The moments $\sum_{n}f_n \int dG \tilde{Psi}(G)*G^j$
are calculated for j=2,4,6. The first corresponds to the
pseudo-kinetic energy, the second term is used for the mass
renormalization with a G-dependent wave function mass.

The wave functions are transformed back into r-space, after truncating
higher Fourier components. 

With the new wave functions we obtain the
projections $\langle\tilde{p}|\tilde\Psi\rangle$. Wave
functions and projections are then renormalized with the factor
$1/\sqrt{
\langle\tilde\Psi|\tilde\Psi\rangle+
\langle\tilde\Psi|\tilde{p}\rangle
O^2\langle\tilde{p}|\tilde\Psi\rangle}$.

With the new wave functions, we calculated pseudo density, one-center
density matrix, and pseudo kinetic energy. This result is obtained
truncation after every grid-point on the radial grid.

For a given cutoff we calculate now the energy contributions.
%
%===============================================================
\section{Testing scattering properties}
%===============================================================
%
We integrate the Schr\"odinger equation with the AE-potential 
outward for a set of energies. For the same energies, we
evaluate the corresponding PAW-equation. This is done with increasing number of
projector functions starting with zero projectors. The pseudo partial
wave is augmented to obtain the corresponding AE partialwaves.

Then we evaluate the phase shift $\Delta_\phi$ at the augmentation radius according to
\begin{displaymath}
\Delta_\phi=\frac{1}{2}-\frac{1}{\pi}\atan(\frac{d\phi/dr}{\phi}+#({\rm nodes})
\end{displaymath}
 where the number of nodes are counted up to the augmentation radius,
 not counting a node at the origin.

The phase shift defined this war has a immediate physical
interpretation. The bonding states defined as $d\phi/dr=0$ at the
radius, lie at $\Delta_{\phi}=n+0.5$, where n is the number of 
nodes, while the antibonding states lie at $\Delta_{\phi}=n+1$.
The hopping parameter is therefore directly proportional to the slope
of the wave function between bonding and antibonding states, and the
band-width is given by the energy difference of the so-obtained
bonding and antibonding states. However, it is important to note that
the band-width depends strongly on the radius chosen for the
construction, which makes physical sense.

In order to make the errors comprehendable, we search for the energy 
displacements $d\epsilon(\epsilon)$ 
that are neccesary to obtain identical phase shifts of
the PAW and the AE calculation, i.e.
\begin{displaymath}
\Delta_{\phi,AE}(\epsilon)=\Delta_{\phi,AE}(\epsilon+d\epsilon)
\end{displaymath}


The results depend strongly on the radius, where the phase shift is
evaluated. I sthe radius large we obtain a step-like phase shifts, and
the energy shifts appear more accurate than in the case of a smaller
radius.



%==========================================================
\subsection{Cutoff convergence of hydrogen}
%==========================================================

We start with the following control file 
\begin{verbatim}
!ACNTL
 !GENERIC ELEMENT='H ' RAUG=3.D0 RBOX=20.D0 !END  
 !DFT     TYPE=1 !END 
 !GRID    R1=1.056E-4 DEX=0.05 NR=250 !END 
 !VALENCE
   !STATE L=0 N=1 F=1. !END 
 !END 
 !VTILDE     TYPE='POLYNOMIAL' RC=0.45 POWER=3 POT(0)=-3.43 !END 
 !COMPENSATE RC=0.1 !END
 !PSCORE     TYPE='POLYNOMIAL' RC=0.45 POWER=2 RHO(0)=0.0 !END 
 !WAVE       L=0 N=1   PSTYPE='HBS' RC=0.40 LAMBDA=6 !END 
 !WAVE       L=0 E=0.  PSTYPE='HBS' RC=0.40 LAMBDA=6 !END 
!END  
!EOB
\begin{verbatim}
and change !WAVE:RC from 0.3 to 0.7.
Fig: h_epw_conv_a.eps


The convergence tests showed that the largest contribution comes from
changing the cutoff radius for the construction of the pseudo partial
waves. Shifting the PS-potential up and down had a also considerable effect
on the convergence, mainly affecting the shape of the dependency. For
a value of about -3.8~H, the convergence was essentially exponential,
while for other values a step occurred, whihc gives the illusion of
convergence at low plane wave cutoffs, but then actually converges
slower than if the value of -3.8~H was chosen. The best convergence
that could be reached left a plane wave cutoff error of 1.8~mH at $E_{PW}=30~Ry$.

The $\lamda=6$ and $r_c=3r_{cov}/4$  was chosen  for the partial waves.
The pseudo-potential used a power of 3 and a cutoff 0.1~a$_0$ smaller
than the cutoff for the wave functions. The ps core density was
constructed with the same cutoff, and a power of 2. We used a potential at the origin

\begin{tabular}{cc}
\hline
Sy & V(0) & $\tilde{n}^c(0) \\
\hline
default & -4.0 & 0.0 \\
H  & -4.0 & 0.0\\
O  & -3.0 & 0.1 \\
\hline
\end{tabular}













